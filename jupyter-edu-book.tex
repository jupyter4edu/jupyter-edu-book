\documentclass[]{book}
\usepackage{lmodern}
\usepackage{amssymb,amsmath}
\usepackage{ifxetex,ifluatex}
\usepackage{fixltx2e} % provides \textsubscript
\ifnum 0\ifxetex 1\fi\ifluatex 1\fi=0 % if pdftex
  \usepackage[T1]{fontenc}
  \usepackage[utf8]{inputenc}
\else % if luatex or xelatex
  \ifxetex
    \usepackage{mathspec}
  \else
    \usepackage{fontspec}
  \fi
  \defaultfontfeatures{Ligatures=TeX,Scale=MatchLowercase}
\fi
% use upquote if available, for straight quotes in verbatim environments
\IfFileExists{upquote.sty}{\usepackage{upquote}}{}
% use microtype if available
\IfFileExists{microtype.sty}{%
\usepackage{microtype}
\UseMicrotypeSet[protrusion]{basicmath} % disable protrusion for tt fonts
}{}
\usepackage[margin=1in]{geometry}
\usepackage{hyperref}
\hypersetup{unicode=true,
            pdftitle={Teaching and Learning with Jupyter},
            pdfauthor={Lorena A. Barba et al.},
            pdfborder={0 0 0},
            breaklinks=true}
\urlstyle{same}  % don't use monospace font for urls
\usepackage{natbib}
\bibliographystyle{apalike}
\usepackage{color}
\usepackage{fancyvrb}
\newcommand{\VerbBar}{|}
\newcommand{\VERB}{\Verb[commandchars=\\\{\}]}
\DefineVerbatimEnvironment{Highlighting}{Verbatim}{commandchars=\\\{\}}
% Add ',fontsize=\small' for more characters per line
\usepackage{framed}
\definecolor{shadecolor}{RGB}{248,248,248}
\newenvironment{Shaded}{\begin{snugshade}}{\end{snugshade}}
\newcommand{\KeywordTok}[1]{\textcolor[rgb]{0.13,0.29,0.53}{\textbf{#1}}}
\newcommand{\DataTypeTok}[1]{\textcolor[rgb]{0.13,0.29,0.53}{#1}}
\newcommand{\DecValTok}[1]{\textcolor[rgb]{0.00,0.00,0.81}{#1}}
\newcommand{\BaseNTok}[1]{\textcolor[rgb]{0.00,0.00,0.81}{#1}}
\newcommand{\FloatTok}[1]{\textcolor[rgb]{0.00,0.00,0.81}{#1}}
\newcommand{\ConstantTok}[1]{\textcolor[rgb]{0.00,0.00,0.00}{#1}}
\newcommand{\CharTok}[1]{\textcolor[rgb]{0.31,0.60,0.02}{#1}}
\newcommand{\SpecialCharTok}[1]{\textcolor[rgb]{0.00,0.00,0.00}{#1}}
\newcommand{\StringTok}[1]{\textcolor[rgb]{0.31,0.60,0.02}{#1}}
\newcommand{\VerbatimStringTok}[1]{\textcolor[rgb]{0.31,0.60,0.02}{#1}}
\newcommand{\SpecialStringTok}[1]{\textcolor[rgb]{0.31,0.60,0.02}{#1}}
\newcommand{\ImportTok}[1]{#1}
\newcommand{\CommentTok}[1]{\textcolor[rgb]{0.56,0.35,0.01}{\textit{#1}}}
\newcommand{\DocumentationTok}[1]{\textcolor[rgb]{0.56,0.35,0.01}{\textbf{\textit{#1}}}}
\newcommand{\AnnotationTok}[1]{\textcolor[rgb]{0.56,0.35,0.01}{\textbf{\textit{#1}}}}
\newcommand{\CommentVarTok}[1]{\textcolor[rgb]{0.56,0.35,0.01}{\textbf{\textit{#1}}}}
\newcommand{\OtherTok}[1]{\textcolor[rgb]{0.56,0.35,0.01}{#1}}
\newcommand{\FunctionTok}[1]{\textcolor[rgb]{0.00,0.00,0.00}{#1}}
\newcommand{\VariableTok}[1]{\textcolor[rgb]{0.00,0.00,0.00}{#1}}
\newcommand{\ControlFlowTok}[1]{\textcolor[rgb]{0.13,0.29,0.53}{\textbf{#1}}}
\newcommand{\OperatorTok}[1]{\textcolor[rgb]{0.81,0.36,0.00}{\textbf{#1}}}
\newcommand{\BuiltInTok}[1]{#1}
\newcommand{\ExtensionTok}[1]{#1}
\newcommand{\PreprocessorTok}[1]{\textcolor[rgb]{0.56,0.35,0.01}{\textit{#1}}}
\newcommand{\AttributeTok}[1]{\textcolor[rgb]{0.77,0.63,0.00}{#1}}
\newcommand{\RegionMarkerTok}[1]{#1}
\newcommand{\InformationTok}[1]{\textcolor[rgb]{0.56,0.35,0.01}{\textbf{\textit{#1}}}}
\newcommand{\WarningTok}[1]{\textcolor[rgb]{0.56,0.35,0.01}{\textbf{\textit{#1}}}}
\newcommand{\AlertTok}[1]{\textcolor[rgb]{0.94,0.16,0.16}{#1}}
\newcommand{\ErrorTok}[1]{\textcolor[rgb]{0.64,0.00,0.00}{\textbf{#1}}}
\newcommand{\NormalTok}[1]{#1}
\usepackage{longtable,booktabs}
\usepackage{graphicx,grffile}
\makeatletter
\def\maxwidth{\ifdim\Gin@nat@width>\linewidth\linewidth\else\Gin@nat@width\fi}
\def\maxheight{\ifdim\Gin@nat@height>\textheight\textheight\else\Gin@nat@height\fi}
\makeatother
% Scale images if necessary, so that they will not overflow the page
% margins by default, and it is still possible to overwrite the defaults
% using explicit options in \includegraphics[width, height, ...]{}
\setkeys{Gin}{width=\maxwidth,height=\maxheight,keepaspectratio}
\IfFileExists{parskip.sty}{%
\usepackage{parskip}
}{% else
\setlength{\parindent}{0pt}
\setlength{\parskip}{6pt plus 2pt minus 1pt}
}
\setlength{\emergencystretch}{3em}  % prevent overfull lines
\providecommand{\tightlist}{%
  \setlength{\itemsep}{0pt}\setlength{\parskip}{0pt}}
\setcounter{secnumdepth}{5}
% Redefines (sub)paragraphs to behave more like sections
\ifx\paragraph\undefined\else
\let\oldparagraph\paragraph
\renewcommand{\paragraph}[1]{\oldparagraph{#1}\mbox{}}
\fi
\ifx\subparagraph\undefined\else
\let\oldsubparagraph\subparagraph
\renewcommand{\subparagraph}[1]{\oldsubparagraph{#1}\mbox{}}
\fi

%%% Use protect on footnotes to avoid problems with footnotes in titles
\let\rmarkdownfootnote\footnote%
\def\footnote{\protect\rmarkdownfootnote}

%%% Change title format to be more compact
\usepackage{titling}

% Create subtitle command for use in maketitle
\newcommand{\subtitle}[1]{
  \posttitle{
    \begin{center}\large#1\end{center}
    }
}

\setlength{\droptitle}{-2em}

  \title{Teaching and Learning with Jupyter}
    \pretitle{\vspace{\droptitle}\centering\huge}
  \posttitle{\par}
    \author{Lorena A. Barba et al.}
    \preauthor{\centering\large\emph}
  \postauthor{\par}
      \predate{\centering\large\emph}
  \postdate{\par}
    \date{2018-11-30}

\usepackage{booktabs}
\usepackage{amsthm}
\makeatletter
\def\thm@space@setup{%
  \thm@preskip=8pt plus 2pt minus 4pt
  \thm@postskip=\thm@preskip
}
\makeatother

\begin{document}
\maketitle

{
\setcounter{tocdepth}{1}
\tableofcontents
}
\chapter{Prerequisites}\label{prerequisites}

This is a \emph{sample} book written in \textbf{Markdown}. You can use
anything that Pandoc's Markdown supports, e.g., a math equation
\(a^2 + b^2 = c^2\).

The \textbf{bookdown} package can be installed from CRAN or Github:

\begin{Shaded}
\begin{Highlighting}[]
\KeywordTok{install.packages}\NormalTok{(}\StringTok{"bookdown"}\NormalTok{)}
\CommentTok{# or the development version}
\CommentTok{# devtools::install_github("rstudio/bookdown")}
\end{Highlighting}
\end{Shaded}

Remember each Rmd file contains one and only one chapter, and a chapter
is defined by the first-level heading \texttt{\#}.

To compile this example to PDF, you need XeLaTeX. You are recommended to
install TinyTeX (which includes XeLaTeX):
\url{https://yihui.name/tinytex/}.

\chapter{Introduction}\label{intro}

\chapter{Placeholder}\label{placeholder}

\chapter{About the Authors}\label{about-the-authors}

\section{Project Lead}\label{project-lead}

\subsection{Lorena A. Barba}\label{lorena-a.-barba}

\begin{itemize}
\tightlist
\item
  George Washington University
\item
  \href{mailto:labarba@email.gwu.edu}{\nolinkurl{labarba@email.gwu.edu}}
\item
  \href{https://twitter.com/LorenaABarba}{@LorenaABarba}
\end{itemize}

Lorena A. Barba is Associate Professor of Mechanical and Aerospace
Engineering at the George Washington University. She adopted Jupyter in
2013 and since then used it in every course she teaches. Her open course
materials are well known and used by thousands of learners:
\href{http://lorenabarba.com/blog/cfd-python-12-steps-to-navier-stokes/}{CFD
Python} and
\href{https://github.com/numerical-mooc/numerical-mooc}{Numerical MOOC}
are the best examples.

\section{Authors at the sprint}\label{authors-at-the-sprint}

\subsection{Lecia J. Barker}\label{lecia-j.-barker}

\begin{itemize}
\tightlist
\item
  University of Colorado Boulder
\item
  \href{mailto:lecia.barker@colorado.edu}{\nolinkurl{lecia.barker@colorado.edu}}
\item
  \href{https://twitter.com/leciab}{@leciab}
\end{itemize}

\href{https://www.colorado.edu/cmci/people/information-science/lecia-barker}{Lecia
Barker} is an Associate Professor and Associate Chair of Undergraduate
Studies in the
\href{https://www.colorado.edu/cmci/infoscience}{Department of
Information Science} at the University of Colorado Boulder. She is also
a Senior Research Scientist for the
\href{https://www.ncwit.org/}{National Center for Women \& IT}. Her
research group is studying the
\href{https://csteachingpractices.wordpress.com/}{diffusion and adoption
of teaching practices} in undergraduate computer science. Lecia holds a
Ph.D.~in Communication from CU Boulder and an MBA in Marketing from San
Diego State University.

\subsection{Douglas Blank}\label{douglas-blank}

\begin{itemize}
\tightlist
\item
  Bryn Mawr College
\item
  \href{mailto:dblank@brynmawr.edu}{\nolinkurl{dblank@brynmawr.edu}}
\item
  \href{https://twitter.com/dougblank}{@dougblank}
\end{itemize}

\href{https://cs.brynmawr.edu/~dblank/}{Douglas Blank} is Associate
Professor in the \href{https://cs.brynmawr.edu/}{Department of Computer
Science} at \href{http://brynmawr.edu/}{Bryn Mawr College}, a small,
all-women's college outside of Philadelphia, PA, USA. He has a joint
Ph.D.~in Cognitive Science and Computer Science from Indiana University,
Bloomington. For over 20 years, Douglas has taught all levels of
Computer Science. For the last 4 years, he has used Jupyter notebooks
exclusively in the classroom. Douglas has published in the areas of
Computer Science Education, Robotics, Artificial Intelligence, and Deep
Learning. He is on the advisory board of
\href{https://www.engage-csedu.org}{Engage-CSEdu.org}, a joint project
between Google and the National Center for Women and Information
Technology (NCWIT). Douglas also writes text and code at his website
\href{http://douglasblank.com}{douglasblank.com}.

\subsection{Jed Brown}\label{jed-brown}

\begin{itemize}
\tightlist
\item
  University of Colorado Boulder
\item
  \href{mailto:jed@jedbrown.org}{\nolinkurl{jed@jedbrown.org}}
\item
  \href{https://twitter.com/five9a2}{@five9a2}
\end{itemize}

\href{https://jedbrown.org/}{Jed Brown} is an Assistant Professor of
Computer Science at the University of Colorado Boulder. He has been
teaching numerical and scientific computing courses using Jupyter
Notebook and nbgrader for three years, and leads a research group that
develops computational methods and community software for computational
science.

\subsection{Allen Downey}\label{allen-downey}

\begin{itemize}
\tightlist
\item
  Olin College
\item
  \href{mailto:downey@allendowney.com}{\nolinkurl{downey@allendowney.com}}
\item
  \href{https://twitter.com/AllenDowney}{@AllenDowney}
\end{itemize}

\href{http://www.allendowney.com/wp/}{Allen Downey} is a professor of
Computer Science at Olin College and the author of a series of
open-source textbooks related to software and data science, including
\emph{Think Python}, \emph{Think Bayes}, and \emph{Think Complexity},
published by O'Reilly Media. These books, and the classes based on them,
use Jupyter notebooks extensively. Prof Downey holds a Ph.D.~in computer
science from U.C. Berkeley, and M.S. and B.S. degrees from MIT.

\subsection{Tim George}\label{tim-george}

\begin{itemize}
\tightlist
\item
  Project Jupyter
\item
  \href{mailto:tgeorgeux@gmail.com}{\nolinkurl{tgeorgeux@gmail.com}}
\end{itemize}

\href{https://www.tgeorgeux.com/}{Timothy George} is the Lead UI/UX
Designer for \href{https://jupyter.org/}{Project Jupyter}, focusing
primarily on JupyterLab. In addition to his formal duties, Tim is also
in working with Jupyter on design strategy, future products, governance,
diversity and inclusion. He studied HCI at UC Irvine's Donald Bren
School of Informatics and Computer Science where he received a Master's
Degree.

\subsection{Lindsey Heagy}\label{lindsey-heagy}

\begin{itemize}
\tightlist
\item
  University of California Berkeley
\item
  \href{mailto:lindseyheagy@gmail.com}{\nolinkurl{lindseyheagy@gmail.com}}
\item
  \href{https://twitter.com/lindsey_jh}{@lindsey\_jh}
\end{itemize}

\href{https://www.lindseyjh.ca/}{Lindsey Heagy} is a Postdoctoral
Researcher at the University of California Berkeley working on Project
Jupyter and Jupyter in the geosciences. She recently completed her PhD
at the University of British Columbia in geophysics. She is a project
leader of \href{http://geosci.xyz}{GeoSci.xyz}, an effort to build
collaborative, interactive, web-based textbooks in the geosciences, and
a core contributor to \href{https://www.simpeg.xyz/}{SimPEG}, an open
source framework for geophysical simulation and inversions. The
GeoSci.xyz project relies heavily on Jupyter for making the content come
to life.

\subsection{Kyle Mandli}\label{kyle-mandli}

\begin{itemize}
\tightlist
\item
  Columbia University
\item
  \href{mailto:kyle.mandli@columbia.edu}{\nolinkurl{kyle.mandli@columbia.edu}}
\item
  \href{https://twitter.com/KyleMandli}{@KyleMandli}
\end{itemize}

\href{https://www.columbia.edu/~ktm2132}{Kyle Mandli} is an Assistant
Professor in the Department of Applied Physics and Applied Mathematics
at Columbia University. He has developed a set of openly available
course notes centered around Jupyter notebooks and uses Jupyter for
homework in conjunction with nbgrader. His other research interests
include development of computational methods for coastal hazards such as
storm surge and tsunamis.

\subsection{Jason Moore}\label{jason-moore}

\begin{itemize}
\tightlist
\item
  University of California, Davis
\item
  \href{mailto:jasonmoore@ucdavis.edu}{\nolinkurl{jasonmoore@ucdavis.edu}}
\item
  \href{https://twitter.com/moorepants}{@moorepants}
\end{itemize}

\href{http://moorepants.info/}{Jason Moore} is an Assistant Teaching
Professor of Mechanical and Aerospace Engineering at the University of
California, Davis. He teaches dynamics and mechanical design related
courses. He utilizes Jupyter notebooks to teach modeling and simulation
and is working on a
\href{https://moorepants.github.io/resonance}{textbook about Mechanical
Vibrations}. He is also a core developer of the
\href{http://sympy.org/}{SymPy} and \href{http://pydy.org/}{PyDy}
projects. Jason has PhD, MSc, and BSc degrees in mechanical engineering
from UC Davis and Old Dominion University.

\subsection{David Lippert}\label{david-lippert}

\begin{itemize}
\tightlist
\item
  Leidos
\end{itemize}

David Lippert is a software engineer at
\href{https://www.leidos.com}{Leidos} in Arlington, Virginia. He
utilizes Jupyter notebooks primarily for exploratory data analysis and
for training and evaluating machine learning algorithms. He has written
Jupyter notebooks to create new Dr.~Seuss sonnets and to evaluate if the
\href{https://www.rottentomatoes.com/about}{Rotten Tomatoes Tomatometer}
can be trusted. He has a BA in computer science from Middlebury College.

\subsection{Kyle Niemeyer}\label{kyle-niemeyer}

\begin{itemize}
\tightlist
\item
  Oregon State University
\item
  \href{mailto:kyle.niemeyer@oregonstate.edu}{\nolinkurl{kyle.niemeyer@oregonstate.edu}}
\item
  \href{https://twitter.com/kyleniemeyer}{@kyleniemeyer}
\end{itemize}

\href{https://niemeyer-research-group.github.io/}{Kyle Niemeyer} is an
Assistant Professor of Mechanical Engineering in the School of
Mechanical, Industrial, and Manufacturing Engineering at Oregon State
University. He teaches courses in numerical and analytical methods for
solving differential equations as well as gas dynamics, and recently
developed a
\href{https://softwaredevengresearch.github.io/syllabus/}{graduate
course on software development for engineering research}. His research
group develops and applies methods for modeling combustion and
chemically reacting fluid flows. He is also on the steering committee of
the \href{https://cantera.org/}{Cantera} open-source project for
chemical kinetics, thermodynamics, and transport processes.

\subsection{Ryan Watkins}\label{ryan-watkins}

\begin{itemize}
\tightlist
\item
  George Washington University
\item
  \href{mailto:rwatkins@gwu.edu}{\nolinkurl{rwatkins@gwu.edu}}
\item
  \href{https://twitter.com/parsingscience}{@parsingscience}
\end{itemize}

\href{https://gsehd.gwu.edu/directory/ryan-watkins}{Ryan Watkins} is a
Professor of Educational Technology at George Washington University in
Washington DC. He leads the
\href{https://go.gwu.edu/phd}{Human-Technology Collaboration (HTC)} PhD
program area, and he teaches courses in needs assessment, instructional
design, and research methods. Ryan's research focuses on how people and
organizations define and assess needs. He is co-host of
\href{https://parsingscience.org/}{Parsing Science}, a podcast where
researchers share the stories behind their science. He also developed
the \href{https://wesharescience.org/}{We Share Science} platform for
sharing video abstracts of research.

\subsection{Richard West}\label{richard-west}

\begin{itemize}
\tightlist
\item
  Northeastern University
\item
  \href{mailto:R.West@northeastern.edu}{\nolinkurl{R.West@northeastern.edu}}
\item
  \href{https://twitter.com/richardhwest}{@richardhwest}
\end{itemize}

\href{https://web.northeastern.edu/comocheng/}{Richard West} is
Associate Professor of Chemical Engineering at Northeastern University
in Boston. He leads a research group in computational modeling for
complex reacting systems like combustion or catalysis. He is a core
member of the \href{https://cantera.org/}{Cantera} open-source project.
As well as in an elective on ``computational modeling in chemical
engineering'', he has integrated Python and Jupyter into core classes on
chemical kinetics and reactor design, at both the undergraduate and
graduate levels. As part of his NSF CAREER award, he is developing
modules to teach students to use Python and SciPy to solve chemical
engineering problems.

\subsection{Elizabeth Wickes}\label{elizabeth-wickes}

\begin{itemize}
\tightlist
\item
  University of Illinois at Urbana-Champaign
\item
  \href{mailto:wickes1@illinois.edu}{\nolinkurl{wickes1@illinois.edu}}
\item
  \href{https://twitter.com/elliewix}{@elliewix}
\end{itemize}

\href{https://ischool.illinois.edu/people/elizabeth-wickes}{Elizabeth
Wickes} is a Lecturer at the School of Information Sciences at the
University of Illinois at Urbana-Champaign. She teaches foundational
programming from an information and data sciences perspective, as well
as other coursework on open data and reproducibility. Her programming
course lectures are written in Jupyter Notebooks and the class is taught
via live coding.

\subsection{Carol Willing}\label{carol-willing}

\begin{itemize}
\tightlist
\item
  Cal Poly San Luis Obispo
\item
  \href{mailto:willingc@gmail.com}{\nolinkurl{willingc@gmail.com}}
\item
  \href{https://twitter.com/WillingCarol}{@WillingCarol}
\end{itemize}

\href{https://www.willingconsulting.com/about/}{Carol Willing} is a
Research Software Engineer at Cal Poly San Luis Obispo working full-time
on \href{https://jupyter.org/}{Project Jupyter}. She is a Python
Software Foundation Fellow and former Director; a Project Jupyter
Steering Council member; and a core developer on CPython and Jupyter.
Carol has an M.S. in Management from MIT and a B.S.E. in Electrical
Engineering from Duke.

\subsection{Michael Zingale}\label{michael-zingale}

\begin{itemize}
\tightlist
\item
  Stony Brook University
\item
  \href{mailto:Michael.Zingale@stonybrook.edu}{\nolinkurl{Michael.Zingale@stonybrook.edu}}
\item
  \href{https://twitter.com/Michael_Zingale}{@Michael\_Zingale}
\end{itemize}

\href{http://www.astro.sunysb.edu/mzingale/}{Michael Zingale} is an
Associate Professor and computational astrophysicist at Stony Brook
University. He has a PhD from University of Chicago (2000). He
frequently teaches
\href{http://bender.astro.sunysb.edu/classes/numerical_methods/}{numerical
methods} and
\href{http://bender.astro.sunysb.edu/classes/python-science/}{Python for
scientific computing} graduate courses, relying on Jupyter notebooks and
python for much of the presentation. He is an advocate for open
educational resources, as a founder of the
\href{https://github.com/Open-Astrophysics-Bookshelf/}{Open Astrophysics
Bookshelf project} where he hosts his
\href{http://bender.astro.sunysb.edu/hydro_by_example/CompHydroTutorial.pdf}{\emph{Introduction
to Computational Astrophysical Hydrodynamics}} text.

\bibliography{book.bib,packages.bib}


\end{document}
